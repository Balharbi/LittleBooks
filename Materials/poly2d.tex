Our general purpose tool for constructing local models of functions of two 
variables is the polynomial.  The point of constructing such models
isn't to capture exactly every aspect of the relationship, but to
build a scaffolding that can be used to analyze and interpret data,
hopefully leading to a better description of the relationship.

We imagine that there is an output that is a function of two inputs:
$f(x,y)$.  The polynomial function that we will use will be
$$ f(x,y) = a_0 + a_1 x + a_2 y + a_3 x y + a_4 x^2 + a_5 y^2$$

Note that the six parameters have been subscripted with a number.  This is just for
convenience in referring to them.  You can call them ``a naught'', ``a
one,'' ``a two'', and so on.  Whatever the names, each of them is just
a scalar.

Depending on the values of the parameters $a_0, a_1, a_2, a_3, a_4,
a_5$, this function can take on all sorts of shapes.  But, in general,
{\bf not all of the terms are needed}.



\begin{description}

\item[$a_0$] The {\bf constant term}.  This sets a typical value of
  $f(x,y)$, but doesn't depend on either $x$ or $y$.  It is almost
  always included by default.


\item[$a_1 x$] The {\bf linear term in $x$}. 
  Produces a simple dependence on the input $x$; if the
  input $x$  changes, then the output $f(x,y)$ will change.

\item[$a_2 y$] Likewise, the {\bf linear term in $y$}. This produces a simple dependence on the input $y$.

\item[$a_4 x^2$] The {\bf quadratic term in $x$} can do two things.  It is
  absolutely needed in the model if there is a maximum or minimum with
  respect to $x$.  But, even if there is no extremum, if there is an
  important change in $\frac{\Delta f}{\Delta x}$ as $x$ changes, then
  there should be this quadratic term.  Example: economists often
  speak of diminishing marginal returns --- doubling the amount of
  investment doesn't lead to a doubling in output per dollar of
  investment.

\item[$a_5 y^2$] The {\bf quadratic term in $y$}. Like the quadratic term in $x$, it's needed for there to be an extremum with
  respect to $y$, or a change in $\frac{\Delta f}{\Delta y}$.

\item[$a_3 x y $] The {\bf interaction term}.  This term expresses how the
  inputs $x$ and $y$ interact: perhaps interfering with one another or
  reinforcing one another.  Whenever the output will depend on $x$
  differently for different values of $y$, or vice versa, there should
  be an interaction term included in the model.  

\end{description} 


Almost always, we include the constant and linear terms in a model,
although we might discover that they are not needed if other terms are
added.  The question is generally whether to include the quadratic and
bilinear terms.  

In order to decide which of these terms to include in a model
$f(x,y)$, it helps to ask the following questions about the quadratic terms and interaction terms:

\begin{enumerate}

\item Is there an extremum with respect to $x$?  That is, holding $y$
  fixed, is there a value of $x$ at which $f(x,y)$ takes on a maximum
  or minimum value?  If there is, you will want to include the
  quadratic term in $x$.


\item If there is an extremum with respect to $x$, does its position
  or magnitude depend on the value of $y$?  If so, include the
  interaction term.

\item If there isn't an extremum with respect to $x$, does the slope
  with respect to $x$ depend on $y$? If so, include the interaction term even though there isn't a quadratic term in $x$.

\item The same questions should be asked with respect to $y$ to decide whether to include the quadratic term in $y$.   

\item Both $x$ and $y$ participate in the interaction term, but sometimes one of the variables gives you a clearer indication that an interaction is important.  Include it if warranted for {\bf either} of the variables $x$ and $y$.





\end{enumerate}


Decide which terms should be included in local models in these
situations:

\begin{description}

\item[Bicycle speed] A bicycle's speed $V$ depends on both the
  steepness $S$ of the terrain and the gear ratio $G$ for the bicycle.
  Assume that the gear ratio is a number between 1 and 6, and let the
  steepness be measured in percent (positive for uphill, negative for
  downhill).  What terms should be included in $V(S,G)$?

\item[Economic production] The output of a factory, $P$, depends both
  on the amount of capital $C$ and the amount of labor $L$.  What
  terms should be included in $P(C,L)$?

\item[Infectious disease] The number of people $N$ who get an illness such
  as the flu depends on both the number of people who already have the
  illness $I$, and the number who are susceptible $S$.  What terms
  should be included in $N(S,I)$?

\item[Survival of chicks] The number of surviving fledglings $F$ of a
  mother bird depends on the number of eggs $N$ that are laid and the time
  that the mother spends collecting food $T$.  What terms should be
  included in $F(N,T)$?

\item[Day length] The length of daylight $D$ depends on both the time
  of the year $M$ (for month) and the latitude $L$.  What terms should
  be included in $D(M,L)$?

\item[Growth of a crop] The yield $Y$ of a crop (bushels/acre) depends
  both on the amount of water applied ($W$, inches/acre) and the amount of
  fertilizer ($F$, lbs/acre).  What terms should be included in
  $Y(W,F)$

\item[Probability of admission to college]  The probability $P$ that an
  applicant will be admitted to college depends on many things, but we
  will restrict consideration here to the math $M$ and verbal $V$
  scores on an entrance examination such as the ACT or SAT.  What
  terms should be included in $P(M,V)$?

\item[School effectiveness]  The effectiveness $E$ of an elementary
  school (perhaps as measured imperfectly by standardized tests)
  depends on both the qualifications of the teachers and the class
  size $S$.  We'll crassly measure the across-the-district teacher
  qualification with the average teacher pay, $P$.  What terms should
  be included in $E(S,P)$?

\end{description}

We can generalize this approach to more than two variables.  Here is a
function of (at least) three-variables.

\begin{description}

\item[Probability of a heart attack] The probability of a heart attack
  $p$ as a function of age $A$ and amount of exercise $E$, and number
  of calories in the diet $C$. What terms should be included in $p(A,E,C)$

\end{description}

